\chapter{序論}\label{abst}
\section{背景と目的}
近年,日常生活の中での手続きの多くがコンピュータにより行われている.
人の負担を軽減するために,時代に応じたシステムが構築されている.
セキュリティ分野においても様々な場所でコンピュータによる個人認証が行われており,
個人認証技術の発達が進んでいる.現在利用されている個人認証システムは,
パスワードやICカードを使用したものが多く見られる.しかし,これらは,パスワードの流出,カードの盗難など
により偽装が比較的容易である.また、複雑なパスワードを設定してしまうと利用者の負担も大きくなる.
そこで現在,多く用いられている個人認証システムとして,バイオメトリクス認証がある.バイオメトリクス認証とは,
指紋,虹彩,顔,座り方,歩き方など個人の特徴をもとに認証を行うシステムである.しかし,指紋や虹彩による認証は,
特別な機器や動作を必要とするため,不便である.一方,顔を用いた認証は特別な機器や動作を必要としない.しかし,
周囲の環境の変化により,認証の精度が低下してしまう問題点がある.

そこで,本研究では,顔などの身体的特徴による個人認証(バイオメトリクス)ではなく,
人間の動作を用いる動作的特徴による個人認証(ソフトバイオメトリクス)に注目した.
人間の動作は人によって異なった癖があり,それにより個人認証ができるのではないかと考えた.
さらに,動作的特徴を用いた認証は,先に述べた身体的特徴と組み合わせて利用することも可能である.
柔軟な応用が可能な点も大きな利点であると考える.実際に,歩き方により個人を認証する技術\cite{rgb_camera}は開発されており,
顔が映らない状態でも,高確率で人物を特定することが可能になっている.本論文では,座り方による個人認証を目的とする.
パソコンを使用するときには座った状態が多いので,座り方での個人認証が有効であると考える.

本論文では,骨格情報を用いた個人認証システムを提案し,骨格情報の取得にはMicrosoft社のKinectを用いて20箇所の関節情報を取得し,
身体的特徴を抽出する.特徴抽出後,サポートベクタマシーン(Support Vector Machine : SVM)を用いて,さまざまな学習方法の比較実験を行う.
そして,本人と本人以外の2クラスに分類を行い,ソフトバイオメトリクスへの応用可能性について検討する.


\section{論文構成}
本論文は全6章で構成されている.第1章では背景と目的について述べた.第2章では骨格情報について,第3章ではSVMについて,
第4章では提案手法について述べる.第5章では実験の結果,および考察を述べ,第6章において結論を述べる.
