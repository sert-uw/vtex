\chapter{関連研究}\label{related_work.tex}

本研究に関連する研究としては, 片桐, 杉村の研究\cite{cite_2}\cite{cite_3}や真部, 菅原の研究\cite{cite_4}が挙げられる. 

片桐, 杉村はペンライトで空中に文字を書き, それをビデオカメラで撮影する手法を提案している. 署名照合には日本サイバーサイン(株)製の商用化されているサイン認証照合エンジン(DPマッチングを使用)を適用し, 本人拒否率(False Rejection Rate : FRR), 他人受入率(False Acceptance Rate : FAR)共に3.6\%での認証が可能であるとしている. そして空中署名による個人認証の有効性を確認するため, 提案手法に加え, カメラの前にガラス板を設置し, その上に署名を行う場合と電子タブレットを用いた場合との比較も行っている. その結果,
\\
\begin{itemize}
  \item{空中署名は運動自由度が大きく真似しにくい}
  \item{他人の署名の形や運動動作を見ても, 手本にならず, 真似しにくい}
  \item{空中署名は運動自由度が大きく, 個人の癖がより顕著に表れる}\\
\end{itemize}
といった考察がなされている.

真部, 菅原の研究では, 空中署名と歩行動作を扱っているが, ここでは空中署名のみについて述べる. 署名の計測装置としてはRGBカメラや深度センサを備えたMicrosoft社製のゲーム用デバイスであるKinectを使用し, ペンやセンサを身に付けることなく指先の3次元空間座標を取得している. 主成分分析により特徴を抽出し, Dynamic Time Warpingによるマッチング偽距離による照合を行い, FRR, FARは共に25.7\%であった.

これらの研究は共に空中署名の利用可能性を示しながらも, 認証精度の点で他のバイオメトリクスの手法には追い付けておらず, 精度の向上が課題である.