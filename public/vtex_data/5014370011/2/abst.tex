\chapter{序論}\label{abst}

\section{背景と目的}

情報社会の発展に伴い, セキュリティシステムを利用する機会が増加している. 実世界, インターネットを含め, 最も多く利用されているのはパスワードに代表される知識ベースやICカード等を用いる物体ベースのシステム群である. これらは導入が容易ではあるが, 忘却や窃盗, 又は偽造(成りすまし)の問題が存在する\cite{cite_1}.

そこで現在, 盛んな研究と実用化が進められているのがバイオメトリクス認証技術である. こちらは認証のための情報として利用者の生体情報を使用する. 個人に固有の情報を用いるため, 他人による偽造が困難であること, 忘却や窃盗の心配が少ないことから, 従来よりも強固なセキュリティシステムの構築が期待される. 表\ref{tab:bio}に示すように, バイオメトリクスは身体的特徴と行動的特徴に分けられる. またこれらとは別にソフトバイオメトリクスと呼ばれる情報も存在する. ソフトバイオメトリクスとは“個人についての特徴であるがどの2人を識別するのにも固有性と普遍性が十分ではない情報”と定義され, 表\ref{tab:soft_bio}のような情報が含まれる. こちらは単体では利用せず, 複数の情報を組み合わせて個人を特定したり, バイオメトリクスと組み合わせることで精度向上・処理時間の削減を目的として利用される.

\begin{table}[htbp]
  \begin{center}
    \caption{バイオメトリクスの分類}
    \begin{tabular}{|c||c|} \hline
      身体的特徴 & 指紋 掌形 網膜 虹彩 顔 静脈 音声 耳形 DNA \\ \hline
      行動的特徴 & 筆跡 キーストローク リップムーブメント まばたき 歩容 \\ \hline
    \end{tabular}
   \label{tab:bio}
  \end{center}
\end{table}

\begin{table}[htbp]
  \begin{center}
    \caption{ソフトバイオメトリクスの例}
    \begin{tabular}{|c|} \hline
      性別 民族 目の色 身長 体重 \\ 
      年齢 服の色 髪の色 歩行速度 \\ \hline
    \end{tabular}
    \label{tab:soft_bio}
  \end{center}
\end{table}

実用化されているバイオメトリクス認証システムでは主に指紋や静脈, 顔, 虹彩を認証情報として利用する\cite{cite_1}. これらは身体的特徴に含まれ, 本人と他人の弁別性が特に高い情報である. 非常に高精度な認証が可能であるが, これらの情報は本人であってもパスワードのように自由な変更が不可能である. そのため, もし一度でも他人に偽造された場合は一生涯に渡ってその情報の再利用が不可能となる. 

そこで, 本研究ではバイオメトリクス認証の手法として空中署名に注目した. これは, 空中に署名を行う動作からその人の癖や特徴を読み取り認証を行うもので, バイオメトリクスの中では行動的特徴に分類される. 行動的特徴によるバイオメトリクス認証は対象動作の再現性や普遍性により, 身体的特徴による認証よりも精度が低下する傾向にある. しかし署名は多くの人にとって長年慣れ親しんだ動作であり, 動作の再現性・普遍性は十分であると考えられる. また, 本研究では署名を紙面上ではなく空中で行うため偽造は困難であるといえるが, 仮に他人に偽造されたとしても署名する文字は自由に変更可能である. これは身体的特徴には無い利点である. さらに, 他の行動的特徴と比較しても動作の変更がより自然に行える点で有利である.

また, 署名による認証は“同じ文字を同じ動作で”署名する必要があり, 署名文字と動作に含まれる癖の両方を用いるといえる. 文字や動作には一定の個人性・普遍性が含まれるが, それ単体では個人の特定は難しい. その意味で, 署名はソフトバイオメトリクスの側面も持つといえる. これは本研究成果の応用可能性として, その他のソフトバイオメトリクス情報との組み合わせにより, マルチモーダルソフトバイオメトリクス認証システムの構築も期待できることを示唆している.

以上より, 空中署名にはバイオメトリクスとして優れた性質があると考えられる. よって本研究では空中署名による個人認証を目指すこととする.

\newpage
\section{論文構成}
本論文は全6章で構成される. 第1章では本研究における背景と目的を述べた. 第2章では署名による個人認証について, 第3章では関連研究を述べる. そして第4章で提案手法を, 第5章にて評価実験を述べた後, 第6章にて結論を述べる.
